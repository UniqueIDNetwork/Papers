\documentclass{article}
\usepackage{url,setspace,amssymb,tikz,algorithmicx,algorithm}


\title{proposal}
\begin{document}
\begin{titlepage}
\begin{center}
\begin{Huge}
UniqueID
\end{Huge}

\vspace{45mm}

\begin{huge}
Decentralized identity system based on biometric data and trust networks
\end{huge}


\vspace{85mm}

\textbf{ March 2018}
\\
\textbf{ uniqueid.network}
\end{center}
\end{titlepage}

\newpage

\begin{abstract}
Salaam.
\end{abstract}

\section{Introduction}

\subsection{Paper organization}



\section{Preliminaries}
\subsection{Bitcoin}
\subsection{Smart Contracts}
\subsection{Decentralized Systems}
\subsection{ Cryptography}


\section{ Problem Specification}
In this section we define our problem formally.

\subsection{Problem Statement}
The simplest formulation of our problems is : \\
\textit{Create a decentralized system for assigning exactly one public key to every human, without violating users privacy. }
\\
This definition can create some debates. First, we must define words like user, human and decentralized, we see this in next section. 

\subsection{Terminology}
\subsubsection{Human}
\subsubsection{Decentralized}
\subsubsection{Incentives}
\subsubsection{Privacy}



\section{ UniqueID Design}
In this section we will explain how UniqueID works and solves decentralized proof-of-unique-human problem. Our system is based on some concepts and components that we will cover them in this chapter in detail. Before that we provide overal flow of the system.

\subsection{Overal Flow}

  Uniquid aims to provide a permanent and reliable identity for everyone without violating his privacy.  Users should submit their biometric data on blockchain and then prove that they have submitted valid data. This proofs are “social” and based on trust: It means you must get some “biometric reproducing certificate” from current accepted users or validators. these validator asks you to “reproduce” your claimed biometric and preventing fake biometrics attacks.
  \\
  There are some questions arising here, such as how users assigned to verifiers, what is the incentives of verifiers for checking others claims and Where is the starting point of the system. for addressing these questions and concepts, we discuss them in depth in following subsections.
\subsection{Biometrics}
Biometric data is how we identify and know "humans". We need permanent, unique and well-studied biometric for our system that storing them on blockchain do not create any privacy concerns. 

TABLE


\subsection{Trustless Setup}
In a system where current users should approve new ones, we should have a mechanism for approving very first users. One method is to begin with suffiecient number of trusted persons and include their biometrics and keys as verified users. The challenge is how to select these persons without any need for central third party? Below we explain three solutions for this issue.

\subsubsection*{Transparent Peer-to-Peer Setup} 
Transparency can be the key for solving trustless setup. There are a community of enthusiastic people around the project that can build healthy setup. Providing relevant information in a public channel, the community invited for joining public party in special place and time. everyone who aims to join sending his proposal including name , biometrics and some identity document -just for transparency purposes. In the party they start to verify each other in a peer-to-peer manner. Because the party is public, everyone can join and there would no monopoly or closed individuals in starting. Also it's near impossible to fake an identity because it can detected and claimed by just one person in a party. There can be a live cast of all of the steps of party in public channel for forward transparency.

\subsubsection*{Trusted Famous People}
There are some people with public profile who have no incentive for sabotaging system, e.g. University Professors, known artists and athletes or popular cafes, usually because of their high reputation among other individuals. Collecting some of these people who wants to join in early verifying, and showing all the informations and profiles can be a starting point for system.

\subsubsection*{ Decentralized CAPTCHA Party}
CAPTCHAs {\ref{•}} are mechanisms for recognizing humans from machines. CAPTCHAs are AI-hard and it takes reasonable time for humans to solve them. Consider we construct some hard CAPTCHA that takes about 2 minutes for an average person to solve. We then organize a CAPTCHA party: in a specific time around the world, based on some common randomness(blockhash) a predefined smart contract starts assigning CAPTCHAs to everyone. 

\subsection{Verification Mechanism}
There are number of verifiers, say 100, in the city. Location and city of every verifier recorded in blockchain. We assume that user's public key is $pk$.
\\

 after one user enters the system with his biometric, he should get certificates from sufficient number of these verifiers(3 is enough here). Assigning verifiers to users done via blockchain with user's location and some common randomness. This randomness can produced with another smart contract, with methods like randHound \ref{} or ...   This randomness,$ R_i$ combined and hashed with user's public key ($SHA256(R_i || pk$) can determine the assignment. \\ 
User then goes to verifier in person and \textbf{reproduce}  his biometric in presence of verifier. Reproducing means sampling biometric data with verifier's biometric device, and then ensuring equality of this sample with previous biometrics on blockchain with some local computation. If verifier ensure that two biometrics are same, he $accepts$ user and issue a certificate with his signature on blockchain. This is one of the most important components of UniqueID, because here verifier can make sure that the biometric is real and belongs to the person who claims it. This will prevent fake biometrics attacks and identity thefts from happening.
\\
After user gaining these certificates, the system automatically accepts him  as a valid and unique person. If only one of verifiers does not issue the certificate, user can ask for a reassignment. This is possible when one of the verifiers is corrupted or has made some error. at the end, if user verified ... (What is the mechanism for punishing this verifier??)
\\
For this procees being meaningful, and no one with many attempts can not break it and assigned to his favourable verifiers we add a barrier in entering the system. There are just three ways you can enter the system :
\begin{itemize}
\item Via $invitation$. Every verified user have two invitations that can give to other public keys and bring them access for submitting their biometric and verification proccess. because there is limited number of verified users, there will be always limited number of verifications (and performing an attack needs large human resources).

\item Via $Stake$. User can put some specified amount of money on stake. If he verified by system, this stake returned to him, if not, he loose it. The amount of stake has direct relation with all unverified stakes in that city, more unverified stake, more stake a user should put.

\item Via $Verifiers$. Every verifier -subject to his reputation and trust- can bring specified number  
\end{itemize}
RANDOMNESS TOKENS disributed , PoS, 

\subsection{Trust Delegation}
Trust is a fundumental key in bringing healthy behavior in our system. Every valid person has one "trust" token and can delegate it to every one. With this non-tradable token one can give his ability of verifying to someone else, because he can not actively verify new users. Thus the verifier with large number of trusts have more verifying power, leading to slightly more tokens he give for every new user.
\\
For more resilience to attacks, we can give verification power to those with more than $x$ trust tokens. this parameter can set based on city populations and maturity of system. This leads to collecting trust tokens by verifiers, but how they can do this? Transparency is the key, Imagine verifier who brings live cast of his place on a youtube chaوnnel, or providing access to his office for everyone (doors are open!). By providing more transparency, more people can trust you.
\\
Also every user must have incentives for : delegating her trust, and delegate it to a correct person. Restricting some important features of system (e.g. Universal Basic Income) to those who delegated their trust is one solution. Also for a corrupted verifier (there is some mechanism for finding corrupt behavior), all his trust get suspended for some period of time.




\subsection{Native Token}
We introduced a native token for our platform, However, For every  token system we should answer to at least 3 questions:
\\
1) What is the nessecity of creating new token, Or why the system needs it?
\\
2) What is the utility of this token for users, Or why it has non-zero value?
\\
3) What is the monetary system, Or how tokens issued and distributed.
\\
\textbf{Nessecity:} First, we need a token for creating incentive mechanisms. every decentralized platform needs sophisticated and well-studied incentives, for encouraging healthy behavior and punishing bad actors. One of the powerful tools for designing mechanism is money, because it is related to person's desirability and profits. In our system verifiers rewarded by tokens every time they verify a new, fresh user. Also new users get some tokens for the first time they enter the system. Also some important applications, such as UBI, needs some form of currency.
\\
\textbf{Utility:} The most obvious usage of such a token is for payments and store of value. Incentives bring new users to the system, and new users bring more acceptability and wider market, projected in token price, bringing new incentives for further users. This positive network-effect, combined with mechanism for more incentivising first movers(but not make them big Whales!) can bring non-zero value for the token.(TODO : This is because of the nature of our system). Also this system can  host  many other applications which need unique and permanent identity, such as digital banking, property and real estate on blockchain, reputatution systems etc. These systems can provide their services on UniquID blockchain, using native token or at least paying fee's for their smart contracts.
\\
\textbf{Monetary system:} Our goal is to achieve an issuance scheme which is fair and keeps value of the token. Our proposal is selling fix amount of tokens (say, $a*x$) in pre-sale (with an Initial Coin Offering  or similar methods) and then issuing $x$ tokens for every new user. This tokens distributed between verifiers of that user and herself. So after $a$ new users, tokens in the hands of users is equal to ICO tokens. This can provide $fairness$ for the system, because initial holders are not giants anymore, and after system became popular(=more users) their influence get reduced. Also initial investors have incentives for investment, because they can buy tokens cheap early. It will be promised to them that their tokens have an influence equal to $a$ users, and they can guess what they earn by estimating value of an identity system with $a$ users. This issuance scheme is linear in number of users, However the utility of users(network effect) is quadratic, so we can predict slight rise in price over time. At least, It's not an super-inflationary system : More user's, more utility, more demand and more supply responing them. Choosing $a$ parameter and schemes for incentivising early users (e.g. changning reward over time) is for further research.\\



Figure for comparing network effect and money supply.

\subsection{Decentralized Governance}
There is no idependent entity for decision-making in decentralized systems, so they face governance problems in many levels. Long discussions on Ethereum DAO fork and Bitcoin Segwit2x proposal are clear examples of this issue. In UniqueID, There is new concept that should be governed as well : human work. There should be some mechanism for detecting and punishing $corrupted$ verifiers.
\\
UniqueID is inherently one-person-one-vote, and this gives us new oppurtunities for shaping our governance model. We introduce some hierarchial representative democratic system, based on three layers of representatives. 
\\
First, small communities of $50-100$ persons choose one person from themselves. They put their trust and vote on him, and can also track his activity and ask him about that. Because they are small group of people, it is expected that they know each other well, and can make nessecary communications between themselves and choose qualified representative. these are layer-1 representatives, and they should know each other and follow system changes and discussions, and also choose layer-2 persons. Every $30-40$ of these layer-1 representatives should gather and choose one representative. And the same for layer-3 : every $20-30$ layer-2 people can choose one layer-3 representative from them. See figure X for visual ...
\\
This system should decide for critical decisions such as choosing parameters and consensus rules. Majority consensus seems a very good method for making decisions in a system with verified identities, but also can cause some problems. Not \textbf{every} decision can be made by a majority (51 percent) of votes. Changing some critical parameter of a system (number of verifiers needed for verifying one person) is a clear example of this. So we need to group desicions based on their importance, and put some thresholds on percentage of votes needed for making decisions in every group (super-majority rules).
\\
Because of low participation, lack of incentive for voting and lack of knowledge for making corret decisions we should give some authority to representatives. But for lowering corruption and collusion, in addition to transparency, every decision needs greater consensus among higher layers. for example, if we need 51 percent of all user votes for some parameter change, we need 68, 85, 95 percent  of votes for layer 1,2,3, respectively.
\\




preventing ads and ... in votes\\ 

bribing is much harder, secure re-vote
\subsection{Implementation}

\subsection{Improvements over verification}
The verification system we purpose face some challanges. Bribing or coordinating some verifiers for creating fake identities is the most important one. Fake-ID prevention is the only thing we design our system for solving that. \\
We can see two directions for solving this attacks. First, Set some sophisticated and hard rules for entering people that guarantee hardness of attacks. Second, design some mechanisms for detecting and removing fake identities and punishing bad verifiers. In this section we explain a mechanism of second kind. \\
We first design a very secure registeration way that there is zero-probablity in entering with fake identity, calling it A-judge. 


\section{Challanges }



\subsection{Geographical Dispersion}

\subsection{Privacy}
\subsubsection*{Trust Tokens}
\subsubsection*{Biometric Data}
\subsubsection*{Secure Logins}

\subsection{Hard Forks}
\subsection{Massive Computation and Storage}
\subsection{Identity Theft or Death}
\subsection{Centralization of Validation Mechanism}
\subsection{Fairness}
\subsection{Lack of Incentive}


\section{Analysis}

 
\section{Applications}

\subsection{Novel mining system}
properly designed, I think this PoH can solve PoS problems with reputation and unforgeable unique signatures...

\subsection{End-to-End Secure Voting}
\subsection{Universal Basic Income}
Universal basic income (UBI) is a kind of welfare system that is based on giving everyone a guaranteed basic income, totally independent of any other income. The idea of national basic income dates back to 18th century but various technical and political issues such as lack of trusted computational infrastructure or a form of trusted universal identity prevented implementation of a UBI system. Though UBI idea originally was presented as a way to save the society from poverty and injustice, many AI experts claim that given the fast and tremendous success of AI technology the world is heading toward a mass unemployment and a UBI system is inevitable and certainly required.\\
UniqueID provides a trustable universal identity and building a UBI system on top of it is just as easy as writing a simple smart-contract on Ethereum network or any other blockchain with access to UniqueID's validated identities. 
\subsection{Reputation Systems}
\subsection{General Framework for Operations}
\subsection{Social Media}
\subsection{Fake News}
\subsection{Rethinking Governance}

\section{ Reviews}
We categorize previously existing ideas in this area in two different categories, first decentralized identity management solutions which aim to provide a decentralized infrastructure to host centralized issued identities, including DID, Sovrin, and uPort. Second one is decentralized identity issuance solutions which aim to create or define a new kind of identity that is decentralized from its origin, including Democracy Earth and Bitnation, Proof-of-Personhood.\\
Our work can be categorized in the later one, decentralized identity issuance systems. We want to emphasize that despite previously known solutions, UniqueID can bring decentralized, permanent, unique identity into reality by utilizing biometric authentication and social smart contracts. \\
In the following section we briefly review some of the mentioned schemes and compare them to UniqueID in measures of political and operational decentralization, sybil attack resistance  permanence of issued identities, and proper governance and incentive mechanisms. 

\subsection{uPort}

uPort is a decentralized infrastructure for claiming identities and receiving verification from other parties in the network. One can use uPort to host and digitize her identity such as national ID card or driving license and get verified by officials or other people for being the owner of the claimed identity.\\
 First of all, uPort is not designed for a decentralized originated identity which is well defined. Due to lack of a general identifier in uPort network, uPort do not provide a direct solution to anti-sybil attack and enabling a universal basic income system. Though uPort provide the possibility to define decentralized originated identities but there is no intrinsic incentive or mechanism to do it.

\subsection{Sovrin}

Sovrin is a protocol and decentralized app based on it’s own blockchain, aiming to create sovereign identity and decentralized trust, it is focused on delivering a kind of identity which is secure, private, and partially provable. In current design of Sovrin system, a trusted set of operations maintain issuing new identities in a semi-decentralized manner. By utilizing its own token, issuers are economically incentivize to participate in system, though there’s no governance mechanism to prevent issuing fake identities and sybil attacks. While Sovrin claim that it will become fully decentralized in future but currently there’s no clear scheme for decentralizing issuing new identities in Sovrin network. 

\subsection{Democracy Earth}
Democracy Earth is going to be a decentralized app mainly concerned with providing a infrastructure for liquid democracy and voting. Naturally it does need a unique identity scheme to prevent sybil attacks. Although currently it is proposed that each new user upload a video from herself in a concrete format to prove her uniqueness but there’s no clear scheme with acceptable error that compare videos and verifier uniqueness of new users. It is also suggested that Democracy Earth utilize other platforms to identify its users.
\subsection{everid}
\subsection{ DID}
\subsection{ID2020}


\end{document}
